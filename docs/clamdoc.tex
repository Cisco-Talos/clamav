%  Clam AntiVirus: User Manual
%
%  Copyright (C) 2014 Cisco Systems, Inc.
%  Copyright (C) 2008-2013 Sourcefire, Inc.
%  Copyright (C) 2002 - 2007 Tomasz Kojm <tkojm*clamav.net>
%  Version 0.2x corrected by Dennis Leeuw <dleeuw*made-it.com>
%  Version 0.80 corrected by Tomasz Papszun <tomek*clamav.net>
%
%  This program is free software; you can redistribute it and/or modify
%  it under the terms of the GNU General Public License as published by
%  the Free Software Foundation; either version 2 of the License, or
%  (at your option) any later version.
%
%  This program is distributed in the hope that it will be useful,
%  but WITHOUT ANY WARRANTY; without even the implied warranty of
%  MERCHANTABILITY or FITNESS FOR A PARTICULAR PURPOSE.  See the
%  GNU General Public License for more details.
%
%  You should have received a copy of the GNU General Public License
%  along with this program; if not, write to the Free Software
%  Foundation, Inc., 51 Franklin Street, Fifth Floor, Boston,
%  MA 02110-1301, USA.

\documentclass[a4paper,titlepage,12pt]{article}
\usepackage{amssymb}
\usepackage{pslatex}
\usepackage[dvips]{graphicx}
\usepackage{wrapfig}
\usepackage{boxedminipage}
\usepackage{url}
\usepackage{fancyhdr}
\usepackage{titlesec}
\addtolength{\hoffset}{-0.5cm}
\addtolength{\textwidth}{1cm}
\date{}

\usepackage{color}
\definecolor{grey1}{gray}{0.8}
\definecolor{grey2}{gray}{0.3}

% Based on Antonina Liedtke's article in Linux+ 6/2003
\def\greyp{%
    \unitlength=1mm%
    \begin{picture}(0,0)
	\put(0,-1.5){\textcolor{grey1}{\rule{13.9cm}{5.3mm}}\textcolor{grey2}%
	{\rule{9mm}{5.3mm}}\hss}
    \end{picture}
}

\pagestyle{fancy}
\fancyhead{}
\fancyfoot{}
\renewcommand{\headrulewidth}{0pt}
\fancyhead[RO]{\textbf{\sffamily{{\textcolor{white}{\thepage}}~}}}
\fancyhead[RE]{\footnotesize{\nouppercase{\rightmark~}}}
\fancyhead[LO]{\footnotesize{\greyp{\nouppercase{\leftmark}}}}


\newcommand{\pl}{\vspace{.3cm}}
\newcommand{\rc}[2]{\textbf{#1: } #2\\[4pt]}
\newcommand{\up}[2]{\textbf{--#1: } #2\\[4pt]}
\newcommand{\email}[1]{\texttt{#1}}
\newcommand{\vbt}[1]{\verb+#1+}
\newcommand{\cons}[1]{\vspace{2mm} \noindent \ovalbox {\sffamily #1}
		      \vspace{2mm}}

\begin{document}
    \setcounter{page}{0}

    \pagestyle{empty}
    \includegraphics[width=353pt]{clam}
    \vspace{3cm}
    \begin{flushright}
	\rule[-1ex]{8cm}{3pt}\\
	\huge Clam AntiVirus 0.98.2\\
	\huge \emph{User Manual}\\
    \end{flushright}

    \newpage
    \pagestyle{fancy}
    \tableofcontents
    \vspace{1.0cm}

    \noindent
    \begin{boxedminipage}[b]{\textwidth}
    ClamAV User Manual,
    \copyright \  2014 Cisco Systems, Inc.
    Authors: Tomasz Kojm\\
    This document is distributed under the terms of the GNU General
    Public License v2.\\

    Clam AntiVirus is free software; you can redistribute it and/or modify
    it under the terms of the GNU General Public License as published by
    the Free Software Foundation; either version 2 of the License, or
    (at your option) any later version.\\

    This program is distributed in the hope that it will be useful,
    but WITHOUT ANY WARRANTY; without even the implied warranty of
    MERCHANTABILITY or FITNESS FOR A PARTICULAR PURPOSE.  See the
    GNU General Public License for more details.\\

    You should have received a copy of the GNU General Public License
    along with this program; if not, write to the Free Software
    Foundation, Inc., 51 Franklin Street, Fifth Floor, Boston,
    MA 02110-1301, USA.
    \end{boxedminipage}

    \vspace{0.3cm}
    \noindent
    \begin{boxedminipage}[b]{\textwidth}
    ClamAV and Clam AntiVirus are trademarks of Cisco Systems, Inc.
    \end{boxedminipage}

    \newpage

    \section{Introduction}
    Clam AntiVirus is an open source (GPL) anti-virus toolkit for UNIX,
    designed especially for e-mail scanning on mail gateways. It provides
    a number of utilities including a flexible and scalable multi-threaded
    daemon, a command line scanner and advanced tool for automatic database
    updates. The core of the package is an anti-virus engine available in a
    form of shared library.

    \subsection{Features}

    \begin{itemize}
	\item{Licensed under the GNU General Public License, Version 2}
	\item{POSIX compliant, portable}
	\item{Fast scanning}
	\item{Supports on-access scanning (Linux only)}
	\item{Detects over 1 million viruses, worms and trojans, including
	      Microsoft Office macro viruses, mobile malware, and other threats}
	\item{Built-in bytecode interpreter allows the ClamAV signature writers
	      to create and distribute very complex detection routines and
	      remotely enhance the scanner's functionality}
	\item{Scans within archives and compressed files (also protects
	      against archive bombs), built-in support includes:
	    \begin{itemize}
		\item Zip (including SFX)
		\item RAR (including SFX)
		\item 7Zip
		\item ARJ (including SFX)
		\item Tar
		\item CPIO
		\item Gzip
		\item Bzip2
                \item DMG
                \item IMG
                \item ISO 9660
                \item PKG
                \item HFS+ partition
                \item HFSX partition
                \item APM disk image
                \item GPT disk image
                \item MBR disk image
                \item XAR
                \item XZ
		\item MS OLE2
		\item MS Cabinet Files (including SFX)
		\item MS CHM (Compiled HTML)
		\item MS SZDD compression format
		\item BinHex
		\item SIS (SymbianOS packages)
		\item AutoIt
		\item InstallShield
	    \end{itemize}}
	\item{Supports Portable Executable (32/64-bit) files compressed or obfuscated with:}
	    \begin{itemize}
		\item AsPack
		\item UPX
		\item FSG
		\item Petite
		\item PeSpin
		\item NsPack
		\item wwpack32
		\item MEW
		\item Upack
		\item Y0da Cryptor
	    \end{itemize}
	\item{Supports ELF and Mach-O files (both 32- and 64-bit)}
	\item{Supports almost all mail file formats}
	\item{Support for other special files/formats includes:}
	    \begin{itemize}
		\item HTML
		\item RTF
		\item PDF
		\item Files encrypted with CryptFF and ScrEnc
		\item uuencode
		\item TNEF (winmail.dat)
	    \end{itemize}
	\item{Advanced database updater with support for scripted updates,
	      digital signatures and DNS based database version queries}
    \end{itemize}

    \subsection{Mailing lists and IRC channel}
    If you have a trouble installing or using ClamAV try asking on our mailing
    lists. There are four lists available:
    \begin{itemize}
    \item \textbf{clamav-announce*lists.clamav.net} - info about new versions,
    moderated\footnote{Subscribers are not allowed to post to the mailing
    list}.
    \item \textbf{clamav-users*lists.clamav.net} - user questions
    \item \textbf{clamav-devel*lists.clamav.net} - technical discussions
    \item \textbf{clamav-virusdb*lists.clamav.net} - database update announcements, moderated
    \end{itemize}
    \noindent You can subscribe and search the mailing list archives at: 
    \url{http://www.clamav.net/contact.html#ml}\\
    Alternatively you can try asking on the \verb+#clamav+ IRC channel - launch
    your favourite irc client and type:
    \begin{verbatim}
	/server irc.freenode.net
	/join #clamav
    \end{verbatim}

    \subsection{Virus submitting}
    If you have got a virus which is not detected by your ClamAV with the latest
    databases, please submit the sample at our website:
    \begin{center}
	\url{http://www.clamav.net/malware-sample}
    \end{center}

    \section{Base package}

    \subsection{Supported platforms}
	\subsubsection{UNIX}
	The most popular UNIX operating systems are supported. Clam AntiVirus 0.9x is
	regularly tested on:
	\begin{itemize}
	    \item{GNU/Linux}
	    \item{Solaris}
	    \item{FreeBSD}
	    \item{OpenBSD} \footnote{Installation from a port is recommended.}
	    \item{Mac OS X}
	\end{itemize}

	\subsubsection{Windows}
	Starting with 0.96 ClamAV builds natively under Visual Studio.

    \subsection{Binary packages}
    You can find the up-to-date list of binary packages at our website:
    \url{http://www.clamav.net/download.html#otherversions}

    \section{Installation}

    \subsection{Requirements}\label{sec:components}
    The following components are required to compile ClamAV under UNIX:
    \footnote{For Windows instructions please see win32/README in the
    main source code directory.}
    \begin{itemize}
	\item zlib and zlib-devel packages
    \item openssl version 0.9.8 or higher and libssl-devel packages
	\item gcc compiler suite (tested with 2.9x, 3.x and 4.x series)\\
	      \textbf{If you are compiling with higher optimization levels
	      than the default one (\hbox{-O2} for gcc), be aware that there
	      have been reports of misoptimizations. The build system of ClamAV
	      only checks for bugs affecting the default settings, it is your
	      responsibility to check that your compiler version doesn't
	      have any bugs.}
    \item GNU make (gmake)
    \end{itemize}
    The following packages are optional but \textbf{highly recommended}:
    \begin{itemize}
	\item bzip2 and bzip2-devel library
        \item libxml2 and libxml2-dev library
	\item \verb+check+ unit testing framework \footnote{See section \ref{unit-testing} on how to run the unit tests}.
    \end{itemize}
    The following packages are optional, but \textbf{required for bytecode JIT support}:
    \footnote{if not available ClamAV will fall back to an interpreter}
    \begin{itemize}
        \item GCC C and C++ compilers (minimum 4.1.3, recommended 4.3.4 or newer)\\
	    the package for these compilers are usually called: gcc, g++, or gcc-c++.
	    \footnote{Note that several versions of GCC have bugs when compiling LLVM, see
		\url{http://llvm.org/docs/GettingStarted.html#brokengcc} for a
		    full list.}
	\item OSX Xcode versions prior to 5.0 use a g++ compiler frontend (llvm-gcc) that is not 
	    compatible with ClamAV JIT. It is recommended to either compile ClamAV JIT with 
	    clang++ or to compile ClamAV without JIT.
	\item A supported CPU for the JIT, either of: X86, X86-64, PowerPC, PowerPC64
    \end{itemize}
    The following packages are optional, but needed for the JIT unit tests:
    \begin{itemize}
     \item GNU Make (version 3.79, recommended 3.81)
     \item Python (version 2.5.4 or newer), for running the JIT unit tests
    \end{itemize}

    \subsection{Installing on shell account}
    To install ClamAV locally on an unprivileged shell account you need not
    create any additional users or groups. Assuming your home directory is
    \verb+/home/gary+ you should build it as follows:
    \begin{verbatim}
	$ ./configure --prefix=/home/gary/clamav --disable-clamav
	$ make; make install
    \end{verbatim}
    To test your installation execute:
    \begin{verbatim}
	$ ~/clamav/bin/freshclam
	$ ~/clamav/bin/clamscan ~
    \end{verbatim}
    The \verb+--disable-clamav+ switch disables the check for existence of
    the \emph{clamav} user and group but \verb+clamscan+ would still require an
    unprivileged account to work in a superuser mode.

    \subsection{Adding new system user and group}
    If you are installing ClamAV for the first time, you have to add a new
    user and group to your system:
    \begin{verbatim}
	# groupadd clamav
	# useradd -g clamav -s /bin/false -c "Clam AntiVirus" clamav
    \end{verbatim}
    Consult a system manual if your OS has not \emph{groupadd} and
    \emph{useradd} utilities. \textbf{Don't forget to lock access to the
    account!}

    \subsection{Compilation of base package}
    Once you have created the clamav user and group, please extract the archive:
    \begin{verbatim}
	$ zcat clamav-x.yz.tar.gz | tar xvf -
	$ cd clamav-x.yz
    \end{verbatim}
    Assuming you want to install the configuration files in /etc, configure
    and build the software as follows:
    \begin{verbatim}
	$ ./configure --sysconfdir=/etc
	$ make
	$ su -c "make install"
    \end{verbatim}
    In the last step the software is installed into the /usr/local directory
    and the config files into /etc. \textbf{WARNING: Never enable the SUID
    or SGID bits for Clam AntiVirus binaries.}

    \subsection{Compilation with clamav-milter enabled}
    libmilter and its development files are required. To enable clamav-milter,
    configure ClamAV with
    \begin{verbatim}
	$ ./configure --enable-milter
    \end{verbatim}
    See section /ref{sec:clamavmilter} for more details on clamav-milter.

    \subsection{Running unit tests}\label{unit-testing}
	ClamAV includes unit tests that allow you to test that the compiled binaries work correctly on your platform.
        \\\\
	The first step is to use your OS's package manager to install the \verb+check+ package. 
	If your OS doesn't have that package, you can download it from \url{http://check.sourceforge.net/}, 
	build it and install it.
        \\\\
	To help clamav's configure script locate \verb+check+, it is recommended that you install \verb+pkg-config+, preferably
	using your OS's package manager, or from \url{http://pkg-config.freedesktop.org}.
        \\\\
	The recommended way to run unit-tests is the following, which ensures you will get an error if unit tests cannot be built:
	\footnote{The configure script in ClamAV automatically enables the unit tests, if it finds the check framework, however it doesn't consider it a fatal error if unit tests cannot be enabled.}
	\begin{verbatim}
	 $ ./configure --enable-check
	 $ make
	 $ make check
	\end{verbatim}
	When \verb+make check+ is finished, you should get a message similar to this:
	\begin{verbatim}
==================
All 8 tests passed
==================
	\end{verbatim}
	If a unit test fails, you get a message similar to the following.
        Note that in older versions of make check may report failures due to
        the absence of optional packages. Please make sure you have the
        latest versions of the components noted in section /ref{sec:components}.
	See the next section on how to report a bug when a unit test fails.
	\begin{verbatim}
========================================
1 of 8 tests failed
Please report to http://bugs.clamav.net/
========================================
	\end{verbatim}
	If unit tests are disabled (and you didn't use --enable-check), you will get this message:
	\begin{verbatim}
*** Unit tests disabled in this build
*** Use ./configure --enable-check to enable them

SKIP: check_clamav
PASS: check_clamd.sh
PASS: check_freshclam.sh
PASS: check_sigtool.sh
PASS: check_clamscan.sh
======================
All 4 tests passed
(1 tests were not run)
======================
	\end{verbatim}
	Running \verb+./configure --enable-check+ should tell you why.

    \subsection{Reporting a unit test failure bug}
	If \verb+make check+ says that some tests failed we encourage you to report a bug on our bugzilla: \url{http://bugs.clamav.net}.
	The information we need is (see also \url{http://www.clamav.net/documentation.html#ins-bugs}):
	\begin{itemize}
	 \item The exact output from \verb+make check+	 
	 \item Output of \verb+uname -mrsp+ 
	 \item your \verb+config.log+	 
	 \item The following files from the \verb+unit_tests/+ directory:
		\begin{itemize}
			\item \verb+test.log+
	 		\item \verb+clamscan.log+
			\item \verb+clamdscan.log+
		\end{itemize}
	 \item \verb+/tmp/clamd-test.log+ if it exists
         \item where and how you installed the check package
	 \item Output of \verb+pkg-config check --cflags --libs+
	 \item Optionally if \verb+valgrind+ is available on your platform, the output of the following:
	 \begin{verbatim}
$ make check
$ CK_FORK=no ./libtool --mode=execute valgrind unit_tests/check-clamav
	 \end{verbatim}	
	\end{itemize}

    \subsection{Obtain Latest ClamAV anti-virus signature databases}
    Before you can run ClamAV in daemon mode (clamd), 'clamdscan',
    or 'clamscan' which is ClamAV's command line virus scanner,
    you must have ClamAV Virus Database (.cvd) file(s) installed
    in the appropriate location on your system.  The default
    location for these database files are /usr/local/share/clamav
    (in Linux/Unix).
    \\\\
    Here is a listing of currently available ClamAV Virus Database Files:
    \begin{itemize}
           \item bytecode.cvd        (signatures to detect bytecode in files)
           \item main.cvd            (main ClamAV virus database file)
           \item daily.cvd           (daily update file for ClamAV virus databases)
           \item safebrowsing.cvd    (virus signatures for safe browsing)
    \end{itemize}
    These files can be downloaded via HTTP from the main ClamAV website
    or via the 'freshclam' utility on a periodic basis.  Using 'freshclam'
    is the preferred method of keeping the ClamAV virus database files
    up to date without manual intervention (see section \ref{conf:freshclam} for
    information on how to configure 'freshclam' for automatic updating and section
    \ref{sec:freshclam} for additional details on freshclam).

    \section{Configuration}
    Before proceeding with the steps below, you should
    run the 'clamconf' command, which gives important information
    about your ClamAV configuration. See section \ref{sec:clamconf} 
    for more details.

    \subsection{clamd}
    Before you start using the daemon you have to edit the configuration file
    (in other case \verb+clamd+ won't run):
    \begin{verbatim}
	$ clamd
	ERROR: Please edit the example config file /etc/clamd.conf.
    \end{verbatim}
    This shows the location of the default configuration file. The format and
    options of this file are fully described in the \emph{clamd.conf(5)}
    manual. The config file is well commented and configuration should be
    straightforward.

    \subsubsection{On-access scanning}
    One of the interesting features of \verb+clamd+ is on-access scanning
    based on fanotify, included in Linux since kernel 2.6.36.
    \textbf{This is not required to run clamd}. At the moment the fanotify header is
    only avaliable for Linux.
    \\\\
    Configure on-access scanning in \verb+clamd.conf+ and read the 
    \ref{On-access} section for on-access scanning usage.

    \subsection{clamav-milter}\label{sec:clamavmilter}
    ClamAV $\ge0.95$ includes a new, redesigned clamav-milter. The most notable
    difference is that the internal mode has been dropped and now a working
    clamd companion is required. The second important difference is that now
    the milter has got its own configuration and log files. 
    \\\\
    To compile ClamAV with the clamav-milter just run \verb+./configure+
    \verb+--enable-milter+ and make as usual. In order to use the 
    '--enable-milter' option with 'configure', your system MUST have the milter 
    library installed.  If you use the '--enable-milter' option without the 
    library being installed, you will most likely see output like this during 
    'configure':
    \begin{verbatim}
        checking for libiconv_open in -liconv... no
        checking for iconv... yes
        checking whether in_port_t is defined... yes
        checking for in_addr_t definition... yes
        checking for mi_stop in -lmilter... no
        checking for library containing strlcpy... no
        checking for mi_stop in -lmilter... no
        configure: error: Cannot find libmilter
    \end{verbatim}
    At which point the 'configure' script will stop processing.
    \\\\
    Please consult your MTA's manual on how to connect ClamAV with the milter.

    \subsection{Testing}
    Try to scan recursively the source directory:
    \begin{verbatim}
	$ clamscan -r -l scan.txt clamav-x.yz
    \end{verbatim}
    It should find some test files in the clamav-x.yz/test directory.
    The scan result will be saved in the \verb+scan.txt+ log file
    \footnote{To get more info on clamscan options run 'man clamscan'}.
    To test \verb+clamd+, start it and use \verb+clamdscan+ (or instead connect
    directly to its socket and run the SCAN command):
    \begin{verbatim}
	$ clamdscan -l scan.txt clamav-x.yz
    \end{verbatim}
    Please note that the scanned files must be accessible by the user running
    \verb+clamd+ or you will get an error.

    \subsection{Setting up auto-updating}\label{conf:freshclam}
    \verb+freshclam+ is the automatic database update tool for Clam AntiVirus.
    It can work in two modes:
    \begin{itemize}
	\item interactive - on demand from command line
	\item daemon - silently in the background
    \end{itemize}
    \verb+freshclam+ is advanced tool: it supports scripted updates (instead
    of transferring the whole CVD file at each update it only transfers the
    differences between the latest and the current database via a special
    script), database version checks through DNS, proxy servers (with
    authentication), digital signatures and various error scenarios.
    \textbf{Quick test: run freshclam (as superuser) with no parameters
    and check the output.} If everything is OK you may create the log file in
    /var/log (owned by \emph{clamav} or another user \verb+freshclam+ will be
    running as):
    \begin{verbatim}
	# touch /var/log/freshclam.log
	# chmod 600 /var/log/freshclam.log
	# chown clamav /var/log/freshclam.log
    \end{verbatim}
    Now you \emph{should} edit the configuration file \verb+freshclam.conf+
    and point the \emph{UpdateLogFile} directive to the log file. Finally, to
    run \verb+freshclam+ in the daemon mode, execute:
    \begin{verbatim}
	# freshclam -d
    \end{verbatim}
    The other way is to use the \emph{cron} daemon. You have to add the
    following line to the crontab of \textbf{root} or \textbf{clamav} user:
    {\small
    \begin{verbatim}
N * * * *	/usr/local/bin/freshclam --quiet
    \end{verbatim}}
    \noindent to check for a new database every hour. \textbf{N should be a
    number between 3 and 57 of your choice. Please don't choose any multiple
    of 10, because there are already too many clients using those time slots.}
    Proxy settings are only configurable via the configuration file and
    \verb+freshclam+ will require strict permission settings for the config
    file when \verb+HTTPProxyPassword+ is turned on.
    \begin{verbatim}
	HTTPProxyServer myproxyserver.com
	HTTPProxyPort 1234
	HTTPProxyUsername myusername
	HTTPProxyPassword mypass
    \end{verbatim}

    \subsubsection{Closest mirrors}
    The \verb+DatabaseMirror+ directive in the config file specifies the
    database server \verb+freshclam+ will attempt (up to \verb+MaxAttempts+
    times) to download the database from. The default database mirror
    is \url{database.clamav.net} but multiple directives are allowed.
    In order to download the database from the closest mirror you should  
    configure \verb+freshclam+ to use \url{db.xx.clamav.net} where xx
    represents your country code. For example, if your server is in "Ascension
    Island" you should have the following lines included in \verb+freshclam.conf+:
    \begin{verbatim}
	DNSDatabaseInfo current.cvd.clamav.net
	DatabaseMirror db.ac.clamav.net
	DatabaseMirror database.clamav.net
    \end{verbatim}
    The second entry acts as a fallback in case the connection to the first
    mirror fails for some reason. The full list of two-letters country codes
    is available at \url{http://www.iana.org/cctld/cctld-whois.htm}

    \subsection{ClamAV Active Malware Report}

    The ClamAV Active Malware Report that was introduced in ClamAV 0.94.1 uses
    freshclam to send summary data to our server about the malware that has
    been detected. This data is then used to generate real-time reports on
    active malware. These reports, along with geographical and historic trends,
    will be published on \url{http://www.clamav.net/}.
    \\\\
    The more data that we receive from ClamAV users, the more reports, and the
    better the quality of the reports, will be. To enable the submission of
    data to us for use in the Active Malware Report, enable
    SubmitDetectionStats in freshclam.conf, and LogTime and LogFile in
    clamd.conf. You should only enable this feature if you're running clamd
    to scan incoming data in your environment.
    \\\\
    The only private data that is transferred is an IP address, which is used
    to create the geographical data. The size of the data that is sent is small;
    it contains just the filename, malware name and time of detection. The data
    is sent in sets of 10 records, up to 50 records per session. For example,
    if you have 45 new records, then freshclam will submit 40; if 78 then it
    will submit the latest 50 entries; and if you have 9 records no statistics
    will be sent.

    \section{Usage}

    \subsection{Clam daemon}\label{clamd}
    \verb+clamd+ is a multi-threaded daemon that uses \emph{libclamav}
    to scan files for viruses. It may work in one or both modes listening on:
    \begin{itemize}
	\item Unix (local) socket
	\item TCP socket
    \end{itemize}
    The daemon is fully configurable via the \verb+clamd.conf+ file
    \footnote{man 5 clamd.conf}. \verb+clamd+ recognizes the following commands:
    \begin{itemize}
	\item \textbf{PING}\\
	    Check the daemon's state (should reply with "PONG").
	\item \textbf{VERSION}\\
	    Print program and database versions.
	\item \textbf{RELOAD}\\
	    Reload the databases.
	\item \textbf{SHUTDOWN}\\
	    Perform a clean exit.
	\item \textbf{SCAN file/directory}\\
	    Scan file or directory (recursively) with archive support
	    enabled (a full path is required).
	\item \textbf{RAWSCAN file/directory}\\
	    Scan file or directory (recursively) with archive and special file
	    support disabled (a full path is required).
	\item \textbf{CONTSCAN file/directory}\\
	    Scan file or directory (recursively) with archive support
	    enabled and don't stop the scanning when a virus is found.
	\item \textbf{MULTISCAN file/directory}\\
	    Scan file in a standard way or scan directory (recursively) using
	    multiple threads (to make the scanning faster on SMP machines).
	\item \textbf{ALLMATCHSCAN file/directory}\\
	    ALLMATCHSCAN works just like SCAN except that it sets a mode
	    where, after finding a virus within a file, continues scanning for
            additional viruses.
	\item \textbf{INSTREAM}\\
	    \emph{It is mandatory to prefix this command with \textbf{n} or
	    \textbf{z}.}\\
	    Scan a stream of data. The stream is sent to clamd in chunks,
	    after INSTREAM, on the same socket on which the command
	    was sent. This avoids the overhead of establishing new TCP
	    connections and problems with NAT. The format of the chunk is:
	    \verb+<length><data>+ where \verb+<length>+ is the size of the
	    following data in bytes expressed as a 4 byte unsigned integer in
	    network byte order and \verb+<data>+ is the actual chunk. Streaming
	    is terminated by sending a zero-length chunk. Note: do not exceed
	    StreamMaxLength as defined in clamd.conf, otherwise clamd will
	    reply with \emph{INSTREAM size limit exceeded} and close the
	    connection.
	\item \textbf{FILDES}\\
	    \emph{It is mandatory to newline terminate this command, or prefix
	    with \textbf{n} or \textbf{z}. This command only works on UNIX
	    domain sockets.}\\
	    Scan a file descriptor. After issuing a FILDES command a subsequent
	    rfc2292/bsd4.4 style packet (with at least one dummy character) is
	    sent to clamd carrying the file descriptor to be scanned inside the
	    ancillary data. Alternatively the file descriptor may be sent in
	    the same packet, including the extra character.
	\item \textbf{STATS}\\
	    \emph{It is mandatory to newline terminate this command, or prefix
	    with \textbf{n} or \textbf{z}, it is recommended to only use the
	    \textbf{z} prefix.}\\
	    On this command clamd provides statistics about the scan queue,
	    contents of scan queue, and memory usage. The exact reply format is
	    subject to changes in future releases.
	\item \textbf{IDSESSION, END}\\
	    \emph{It is mandatory to prefix this command with \textbf{n} or
	    \textbf{z}, also all commands inside \textbf{IDSESSION} must be
	    prefixed.}\\
	    Start/end a clamd session. Within a session multiple
	    SCAN, INSTREAM, FILDES, VERSION, STATS commands can be sent on the
	    same socket without opening new connections. Replies from clamd
	    will be in the form \verb+<id>: <response>+ where \verb+<id>+ is
	    the request number (in ASCII, starting from 1) and \verb+<response>+
	    is the usual clamd reply. The reply lines have the same delimiter
	    as the corresponding command had. Clamd will process the commands
	    asynchronously, and reply as soon as it has finished processing.
	    Clamd requires clients to read all the replies it sent, before
	    sending more commands to prevent send() deadlocks. The recommended
	    way to implement a client that uses IDSESSION is with non-blocking
	    sockets, and a select()/poll() loop: whenever send would block,
	    sleep in select/poll until either you can write more data, or read
	    more replies. \emph{Note that using non-blocking sockets without
	    the select/poll loop and alternating recv()/send() doesn't comply
	    with clamd's requirements.} If clamd detects that a client has
	    deadlocked, it will close the connection. Note that clamd may
	    close an IDSESSION connection too if the client doesn't follow the
	    protocol's requirements.
	\item \textbf{STREAM} (deprecated, use \textbf{INSTREAM} instead)\\
	    Scan stream: clamd will return a new port number you should
	    connect to and send data to scan.
    \end{itemize}
    It's recommended to prefix clamd commands with the letter \textbf{z}
    (eg. zSCAN) to indicate that the command will be delimited by a NULL
    character and that clamd should continue reading command data until a NULL
    character is read. The null delimiter assures that the complete command
    and its entire argument will be processed as a single command. Alternatively
    commands may be prefixed with the letter \textbf{n} (e.g. nSCAN) to use
    a newline character as the delimiter. Clamd replies will honour the
    requested terminator in turn. If clamd doesn't recognize the command, or
    the command doesn't follow the requirements specified below, it will reply
    with an error message, and close the connection.
    \noindent
    Clamd can handle the following signals:
    \begin{itemize}
	\item \textbf{SIGTERM} - perform a clean exit
	\item \textbf{SIGHUP} - reopen the log file
	\item \textbf{SIGUSR2} - reload the database
    \end{itemize}
    Clamd should not be started in the background using the shell operator
    \verb+&+ or external tools. Instead, you should run and wait for clamd
    to load the database and daemonize itself. After that, clamd is instantly
    ready to accept connections and perform file scanning.

    \subsection{Clam\textbf{d}scan}
    \verb+clamdscan+ is a simple \verb+clamd+ client. In many cases you can
    use it as a \verb+clamscan+ replacement however you must remember that:
    \begin{itemize}
	\item it only depends on \verb+clamd+
	\item although it accepts the same command line options as
	      \verb+clamscan+ most of them are ignored because they must be
	      enabled directly in \verb+clamd+, i.e. \verb+clamd.conf+
	\item in TCP mode scanned files must be accessible for \verb+clamd+,
	      if you enabled LocalSocket in clamd.conf then clamdscan will
	      try to workaround this limitation by using FILDES
    \end{itemize}

    \subsection{On-access Scanning}\label{On-access}
    There is a special thread in \verb+clamd+ that performs on-access
    scanning under Linux and shares internal virus database
    with the daemon. \textbf{You must follow some important rules when
    using it:}
    \begin{itemize}
	\item Always stop the daemon cleanly - using the SHUTDOWN command or
	      the\\ SIGTERM signal. In other case you can lose access
	      to protected files until the system is restarted.
	\item Never protect the directory your mail-scanner software
	      uses for attachment unpacking. Access to all infected
	      files will be automatically blocked and the scanner (including
	      \verb+clamd+!) will not be able to detect any viruses. In the
	      result \textbf{all infected mails may be delivered.}
    \end{itemize}
    For example, to protect the whole system add the following lines to
    \verb+clamd.conf+:
    \begin{verbatim}
	ScanOnAccess yes
	OnAccessIncludePath /
	OnAccessExcludePath /proc
	OnAccessExcludePath /temporary/dir/of/your/mail/scanning/software
    \end{verbatim}
    For more configuration options, type 'man clamd.conf' or reference the
    example clamd.conf.

    \subsection{Clamdtop}
    \verb+clamdtop+ is a tool to monitor one or multiple instances of clamd.  It
    has a (color) ncurses interface, that shows the jobs in clamd's queue,
    memory usage, and information about the loaded signature database.
    You can specify on the command-line to which clamd(s) it should connect
    to. By default it will attempt to connect to the local clamd as defined
    in clamd.conf.
    \\\\
    For more detailed help, type 'man clamdtop' or 'clamdtop --help'.

    \subsection{Clamscan}
    \verb+clamscan+ is ClamAV's command line virus scanner.  It can be used to
    scan files and/or directories for viruses.  In order for clamscan
    to work proper, the ClamAV virus database files must be installed on
    the system you are using clamscan on.
    \\\\
    The general usage of clamscan is: clamscan [options] [file/directory/-]
    \\\\
    For more detailed help, type 'man clamscan' or 'clamscan --help'.

    \subsection{ClamBC}
    \verb+clambc+ is Clam Anti-Virus' bytecode testing tool.  It can be
    used to test files which contain bytecode. For more detailed help,
    type 'man clambc' or 'clambc --help'.

    \subsection{Freshclam}\label{sec:freshclam}
    \verb+freshclam+ is ClamAV's virus database update tool and reads it's
    configuration from the file 'freshclam.conf' (this may be
    overriden by command line options). Freshclam's default behavior is to
    attempt to update databases that are paired with downloaded cdiffs.
    Potentially corrupted databases are not updated and are automatically
    fully replaced after several failed attempts unless otherwise specified.
    \\\\
    Here is a sample usage including cdiffs:
{\footnotesize
    \begin{verbatim}
$ freshclam

ClamAV update process started at Mon Oct  7 08:15:10 2013
main.cld is up to date (version: 55, sigs: 2424225, f-level: 60, builder: neo)
Downloading daily-17945.cdiff [100%]
Downloading daily-17946.cdiff [100%]
Downloading daily-17947.cdiff [100%]
daily.cld updated (version: 17947, sigs: 406951, f-level: 63, builder: neo)
Downloading bytecode-227.cdiff [100%]
Downloading bytecode-228.cdiff [100%]
bytecode.cld updated (version: 228, sigs: 43, f-level: 63, builder: neo)
Database updated (2831219 signatures) from database.clamav.net (IP: 64.6.100.177)
    \end{verbatim}
}
    For more detailed help, type 'man clamscan' or 'clamscan --help'.

    \subsection{Clamconf}\label{sec:clamconf}
    \verb+clamconf+ is the Clam Anti-Virus configuration utility.  It is used
    for displaying values of configurations options in ClamAV, which
    will show the contents of clamd.conf (or tell you if it is not
    properly configured), the contents of freshclam.conf, and display
    information about software settings, database, platform, and build
    information. Here is a sample clamconf output:
{\footnotesize
    \begin{verbatim}
$ clamconf

Checking configuration files in /etc/clamav

Config file: clamd.conf
-----------------------
ERROR: Please edit the example config file /etc/clamav/clamd.conf

Config file: freshclam.conf
---------------------------
ERROR: Please edit the example config file /etc/clamav/freshclam.conf

clamav-milter.conf not found

Software settings
-----------------
Version: 0.98.2
Optional features supported: MEMPOOL IPv6 AUTOIT_EA06 BZIP2 RAR JIT

Database information
--------------------
Database directory: /xclam/gcc/release/share/clamav
WARNING: freshclam.conf and clamd.conf point to different database directories
print_dbs: Can't open directory /xclam/gcc/release/share/clamav

Platform information
--------------------
uname: Linux 3.5.0-44-generic #67~precise1-Ubuntu SMP Wed Nov 13 16:20:03 UTC 2013 i686
OS: linux-gnu, ARCH: i386, CPU: i686
Full OS version: Ubuntu 12.04.3 LTS
zlib version: 1.2.3.4 (1.2.3.4), compile flags: 55
Triple: i386-pc-linux-gnu
CPU: i686, Little-endian
platform id: 0x0a114d4d0404060401040604

Build information
-----------------
GNU C: 4.6.4 (4.6.4)
GNU C++: 4.6.4 (4.6.4)
CPPFLAGS:
CFLAGS: -g -O0 -D_LARGEFILE_SOURCE -D_LARGEFILE64_SOURCE
CXXFLAGS:
LDFLAGS:
Configure: '--prefix=/xclam/gcc/release/' '--disable-clamav' '--enable-debug' 'CFLAGS=-g -O0'
sizeof(void*) = 4
Engine flevel: 77, dconf: 77
 \end{verbatim}
}
    For more detailed help, type 'man clamconf' or 'clamconf --help'.

    \subsection{Output format}

    \subsubsection{clamscan}
    \verb+clamscan+ writes all regular program messages to \textbf{stdout} and
    errors/warnings to \textbf{stderr}. You can use the option \verb+--stdout+
    to redirect all program messages to \textbf{stdout}. Warnings and error
    messages from \verb+libclamav+ are always printed to \textbf{stderr}.
    A typical output from \verb+clamscan+ looks like this:
    \begin{verbatim}
	/tmp/test/removal-tool.exe: Worm.Sober FOUND
	/tmp/test/md5.o: OK
	/tmp/test/blob.c: OK
	/tmp/test/message.c: OK
	/tmp/test/error.hta: VBS.Inor.D FOUND
    \end{verbatim}
    When a virus is found its name is printed between the \verb+filename:+ and
    \verb+FOUND+ strings. In case of archives the scanner depends on libclamav
    and only prints the first virus found within an archive:
    \begin{verbatim}
	$ clamscan malware.zip 
	malware.zip: Worm.Mydoom.U FOUND
    \end{verbatim}
    When using the --allmatch(-z) flag, clamscan may print multiple virus
    \verb+FOUND+ lines for archives and files.

    \subsubsection{clamd}
    The output format of \verb+clamd+ is very similar to \verb+clamscan+.
    \begin{verbatim}
	$ telnet localhost 3310
	Trying 127.0.0.1...
	Connected to localhost.
	Escape character is '^]'.
	SCAN /home/zolw/test
	/home/zolw/test/clam.exe: ClamAV-Test-File FOUND
	Connection closed by foreign host.
    \end{verbatim}
    In the \textbf{SCAN} mode it closes the connection when the first virus
    is found.
    \begin{verbatim}
	SCAN /home/zolw/test/clam.zip
	/home/zolw/test/clam.zip: ClamAV-Test-File FOUND
    \end{verbatim}
    \textbf{CONTSCAN} and \textbf{MULTISCAN} don't stop scanning in case
    a virus is found.\\
    Error messages are printed in the following format:
    \begin{verbatim}
	SCAN /no/such/file
	/no/such/file: Can't stat() the file. ERROR
    \end{verbatim}

    \section{LibClamAV}
    Libclamav provides an easy and effective way to add a virus protection into
    your software. The library is thread-safe and transparently recognizes and
    scans within archives, mail files, MS Office document files, executables
    and other special formats.

    \subsection{Licence}
    Libclamav is licensed under the GNU GPL v2 licence. This means you are
    \textbf{not allowed} to link commercial, closed-source software
    against it. All software using libclamav must be GPL compliant.

    \subsection{Supported formats and features}

    \subsubsection{Executables}
    The library has a built-in support for 32- and 64-bit Portable Executable,
    ELF and Mach-O files. Additionally, it can handle PE files compressed or
    obfuscated with the following tools:
    \begin{itemize}
	\item Aspack (2.12)
	\item UPX (all versions)
	\item FSG (1.3, 1.31, 1.33, 2.0)
	\item Petite (2.x)
	\item PeSpin (1.1)
	\item NsPack
	\item wwpack32 (1.20)
	\item MEW
	\item Upack
	\item Y0da Cryptor (1.3)
    \end{itemize}

    \subsubsection{Mail files}
    Libclamav can handle almost every mail file format including TNEF
    (winmail.dat) attachments.

    \subsubsection{Archives and compressed files}
    The following archive and compression formats are supported by internal
    handlers:
    \begin{itemize}
	\item Zip (+ SFX)
	\item RAR (+ SFX)
	\item 7Zip
	\item Tar
	\item CPIO
	\item Gzip
	\item Bzip2
        \item DMG
        \item IMG
        \item ISO 9660
        \item PKG
        \item HFS+ partition
        \item HFSX partition
        \item APM disk image
        \item GPT disk image
        \item MBR disk image
        \item XAR
        \item XZ
	\item MS OLE2
	\item MS Cabinet Files (+ SFX)
	\item MS CHM (Compiled HTML)
	\item MS SZDD compression format
	\item BinHex
	\item SIS (SymbianOS packages)
	\item AutoIt
	\item NSIS
	\item InstallShield
    \end{itemize}

    \subsubsection{Documents}
    The most popular file formats are supported:
    \begin{itemize}
	\item MS Office and MacOffice files
	\item RTF
	\item PDF
	\item HTML
    \end{itemize}
    In the case of Office, RTF and PDF files, libclamav will only extract the
    embedded objects and will not decode the text data itself. The text
    decoding and normalization is only performed for HTML files.

    \subsubsection{Data Loss Prevention}
    Libclamav includes a DLP module which can detect the following
    credit card issuers: AMEX, VISA, MasterCard, Discover, Diner's Club,
    and JCB and U.S. social security numbers inside text files.
    \\\\
    Future versions of Libclamav may include additional features to
    detect other credit cards and other forms of PII (Personally
    Identifiable Information) which may be transmitted without the
    benefit of being encrypted.

    \subsubsection{Others}
    Libclamav can handle various obfuscators, encoders, files vulnerable to
    security risks such as:
    \begin{itemize}
	\item JPEG (exploit detection)
	\item RIFF (exploit detection)
	\item uuencode
	\item ScrEnc obfuscation
	\item CryptFF
    \end{itemize}

    \subsection{API}

    \subsubsection{Header file}
    Every program using libclamav must include the header file \verb+clamav.h+:
    \begin{verbatim}
	#include <clamav.h>
    \end{verbatim}

    \subsubsection{Initialization}
    Before using libclamav, you should call \verb+cl_init()+ to initialize
    it. When it's done, you're ready to create a new scan engine by calling
    \verb+cl_engine_new()+. To free resources allocated by the engine use
    \verb+cl_engine_free()+. Function prototypes:
    \begin{verbatim}
	int cl_init(unsigned int options);
	struct cl_engine *cl_engine_new(void);
	int cl_engine_free(struct cl_engine *engine);
    \end{verbatim}
    \verb+cl_init()+ and \verb+cl_engine_free()+ return \verb+CL_SUCCESS+
    on success or another code on error. \verb+cl_engine_new()+ return
    a pointer or NULL if there's not enough memory to allocate a new
    engine structure.

    \subsubsection{Database loading}
    The following set of functions provides an interface for loading
    the virus database:
    \begin{verbatim}
	const char *cl_retdbdir(void);

	int cl_load(const char *path, struct cl_engine *engine,
		    unsigned int *signo, unsigned int options);
    \end{verbatim}
    \verb+cl_retdbdir()+ returns the default (hardcoded) path to the directory
    with ClamAV databases.
    \verb+cl_load()+ loads a single database file or all databases from a
    given directory (when \verb+path+ points to a directory). The second
    argument is used for passing in the pointer to the engine that should
    be previously allocated with \verb+cl_engine_new()+. A number of loaded
    signatures will be \textbf{added} to \verb+signo+ \footnote{Remember to
    initialize the virus counter variable with 0.}. The last argument can
    pass the following flags:
    \begin{itemize}
	\item \textbf{CL\_DB\_STDOPT}\\
	This is an alias for a recommended set of scan options.
	\item \textbf{CL\_DB\_PHISHING}\\
	Load phishing signatures.
	\item \textbf{CL\_DB\_PHISHING\_URLS}\\
	Initialize the phishing detection module and load .wdb and .pdb files.
	\item \textbf{CL\_DB\_PUA}\\
	Load signatures for Potentially Unwanted Applications.
	\item \textbf{CL\_DB\_OFFICIAL\_ONLY}\\
	Only load official signatures from digitally signed databases.
	\item \textbf{CL\_DB\_BYTECODE}\\
	Load bytecode.
    \end{itemize}
    \verb+cl_load()+ returns \verb+CL_SUCCESS+ on success and another code on
    failure.
    \begin{verbatim}
	    ...
	    struct cl_engine *engine;
	    unsigned int sigs = 0;
	    int ret;

	if((ret = cl_init()) != CL_SUCCESS) {
	    printf("cl_init() error: %s\n", cl_strerror(ret));
	    return 1;
	}

	if(!(engine = cl_engine_new())) {
	    printf("Can't create new engine\n");
	    return 1;
	}

	ret = cl_load(cl_retdbdir(), engine, &sigs, CL_DB_STDOPT);
    \end{verbatim}

    \subsubsection{Error handling}
    Use \verb+cl_strerror()+ to convert error codes into human readable
    messages.  The function returns a statically allocated string:
    \begin{verbatim}
	if(ret != CL_SUCCESS) {
	    printf("cl_load() error: %s\n", cl_strerror(ret));
	    cl_engine_free(engine);
	    return 1;
	}
    \end{verbatim}

    \subsubsection{Engine structure}
    When all required databases are loaded you should prepare the detection
    engine by calling \verb+cl_engine_compile()+. In case of failure you
    should still free the memory allocated to the engine with
    \verb+cl_engine_free()+:
    \begin{verbatim}
	int cl_engine_compile(struct cl_engine *engine);
    \end{verbatim}
    In our example:
    \begin{verbatim}
	if((ret = cl_engine_compile(engine)) != CL_SUCCESS) {
	    printf("cl_engine_compile() error: %s\n", cl_strerror(ret));
	    cl_engine_free(engine);
	    return 1;
	}
    \end{verbatim}

    \subsubsection{Limits}
    When you create a new engine with \verb+cl_engine_new()+, it will have
    all internal settings set to default values as recommended by the
    ClamAV authors. It's possible to check and modify the values (numerical
    and strings) using the following set of functions:
    \begin{verbatim}
int cl_engine_set_num(struct cl_engine *engine,
  enum cl_engine_field field, long long num);

long long cl_engine_get_num(const struct cl_engine *engine,
  enum cl_engine_field field, int *err);

int cl_engine_set_str(struct cl_engine *engine,
  enum cl_engine_field field, const char *str);

const char *cl_engine_get_str(const struct cl_engine *engine,
  enum cl_engine_field field, int *err);
    \end{verbatim}
    Please don't modify the default values unless you know what you're doing.
    Refer to the ClamAV sources (clamscan, clamd) for examples.

    \subsubsection{Database checks}
    It's very important  to keep the internal instance of the database up to
    date. You can watch database changes with the \verb+cl_stat..()+ family
    of functions.
    \begin{verbatim}
	int cl_statinidir(const char *dirname, struct cl_stat *dbstat);
	int cl_statchkdir(const struct cl_stat *dbstat);
	int cl_statfree(struct cl_stat *dbstat);
    \end{verbatim}
    Initialization:
    \begin{verbatim}
	    ...
	    struct cl_stat dbstat;

	memset(&dbstat, 0, sizeof(struct cl_stat));
	cl_statinidir(dbdir, &dbstat);
    \end{verbatim}
    To check for a change you just need to call \verb+cl_statchkdir+ and check
    its return value (0 - no change, 1 - some change occured). Remember to reset
    the \verb+cl_stat+ structure after reloading the database.
    \begin{verbatim}
	if(cl_statchkdir(&dbstat) == 1) {
	    reload_database...;
	    cl_statfree(&dbstat);
	    cl_statinidir(cl_retdbdir(), &dbstat);
	}
    \end{verbatim}
    Libclamav $\ge0.96$ includes and additional call to check the number of
    signatures that can be loaded from a given directory:
    \begin{verbatim}
	int cl_countsigs(const char *path, unsigned int countoptions,
	    unsigned int *sigs);
    \end{verbatim}
    The first argument points to the database directory, the second one
    specifies what signatures should be counted:
    \verb+CL_COUNTSIGS_OFFICIAL+ (official signatures),\\
    \verb+CL_COUNTSIGS_UNOFFICIAL+ (third party signatures),
    \verb+CL_COUNTSIGS_ALL+ (all signatures). The last argument points
    to the counter to which the number of detected signatures will
    be added (therefore the counter should be initially set to 0).
    The call returns \verb+CL_SUCCESS+ or an error code.

    \subsubsection{Data scan functions}
    It's possible to scan a file or descriptor using:
    \begin{verbatim}
	int cl_scanfile(const char *filename, const char **virname,
	unsigned long int *scanned, const struct cl_engine *engine,
	unsigned int options);

	int cl_scandesc(int desc, const char **virname, unsigned
	long int *scanned, const struct cl_engine *engine,
	unsigned int options);
    \end{verbatim}
    Both functions will store a virus name under the pointer \verb+virname+,
    the virus name is part of the engine structure and must not be released
    directly. If the third argument (\verb+scanned+) is not NULL, the
    functions will increase its value with the size of scanned data (in
    \verb+CL_COUNT_PRECISION+ units).
    The last argument (\verb+options+) specified the scan options and supports
    the following flags (which can be combined using bit operators):
    \begin{itemize}
	\item \textbf{CL\_SCAN\_STDOPT}\\
	      This is an alias for a recommended set of scan options. You
	      should use it to make your software ready for new features
	      in the future versions of libclamav.
	\item \textbf{CL\_SCAN\_RAW}\\
	      Use it alone if you want to disable support for special files.
	\item \textbf{CL\_SCAN\_ARCHIVE}\\
	      This flag enables transparent scanning of various archive formats.
	\item \textbf{CL\_SCAN\_BLOCKENCRYPTED}\\
	      With this flag the library will mark encrypted archives as viruses
	      (Encrypted.Zip, Encrypted.RAR).
	\item \textbf{CL\_SCAN\_MAIL}\\
	      Enable support for mail files.
	\item \textbf{CL\_SCAN\_OLE2}\\
	      Enables support for OLE2 containers (used by MS Office and .msi
	      files).
	\item \textbf{CL\_SCAN\_PDF}\\
	      Enables scanning within PDF files.
	\item \textbf{CL\_SCAN\_SWF}\\
	      Enables scanning within SWF files, notably compressed SWF.
	\item \textbf{CL\_SCAN\_PE}\\
	      This flag enables deep scanning of Portable Executable files and
	      allows libclamav to unpack executables compressed with run-time
	      unpackers.
	\item \textbf{CL\_SCAN\_ELF}\\
	      Enable support for ELF files.
	\item \textbf{CL\_SCAN\_BLOCKBROKEN}\\
	      libclamav will try to detect broken executables and mark them as
	      Broken.Executable.
	\item \textbf{CL\_SCAN\_HTML}\\
	      This flag enables HTML normalisation (including ScrEnc
	      decryption).
	\item \textbf{CL\_SCAN\_ALGORITHMIC}\\
	      Enable algorithmic detection of viruses.
	\item \textbf{CL\_SCAN\_PHISHING\_BLOCKSSL}\\
	      Phishing module: always block SSL mismatches in URLs.
	\item \textbf{CL\_SCAN\_PHISHING\_BLOCKCLOAK}\\
	      Phishing module: always block cloaked URLs.
	\item \textbf{CL\_SCAN\_STRUCTURED}\\
	      Enable the DLP module which scans for credit card and SSN
	      numbers.
	\item \textbf{CL\_SCAN\_STRUCTURED\_SSN\_NORMAL}\\
	      Search for SSNs formatted as xx-yy-zzzz.
	\item \textbf{CL\_SCAN\_STRUCTURED\_SSN\_STRIPPED}\\
	      Search for SSNs formatted as xxyyzzzz.
	\item \textbf{CL\_SCAN\_PARTIAL\_MESSAGE}\\
	      Scan RFC1341 messages split over many emails. You will need to
	      periodically clean up \verb+$TemporaryDirectory/clamav-partial+
	      directory.
	\item \textbf{CL\_SCAN\_HEURISTIC\_PRECEDENCE}\\
	      Allow heuristic match to take precedence. When enabled, if
	      a heuristic scan (such as phishingScan) detects a possible
	      virus/phish it will stop scan immediately. Recommended, saves CPU
	      scan-time. When disabled, virus/phish detected by heuristic scans
	      will be reported only at the end of a scan. If an archive
	      contains both a heuristically detected virus/phishing, and a real
	      malware, the real malware will be reported.
	\item \textbf{CL\_SCAN\_BLOCKMACROS}\\
	      OLE2 containers, which contain VBA macros will be marked infected
	      (Heuristics.OLE2.ContainsMacros).
    \end{itemize}
    All functions return \verb+CL_CLEAN+ when the file seems clean,
    \verb+CL_VIRUS+ when a virus is detected and another value on failure.
    \begin{verbatim}
	    ...
	    const char *virname;

	if((ret = cl_scanfile("/tmp/test.exe", &virname, NULL, engine,
	CL_SCAN_STDOPT)) == CL_VIRUS) {
	    printf("Virus detected: %s\n", virname);
	} else {
	    printf("No virus detected.\n");
	    if(ret != CL_CLEAN)
	        printf("Error: %s\n", cl_strerror(ret));
	}
    \end{verbatim}

    \subsubsection{Memory}
    Because the engine structure occupies a few megabytes of system memory, you
    should release it with \verb+cl_engine_free()+ if you no longer need to
    scan files.

    \subsubsection{Forking daemons}
    If you're using libclamav with a forking daemon you should call
    \verb+srand()+ inside a forked child before making any calls to the
    libclamav functions. This will avoid possible collisions with temporary
    filenames created by other processes of the daemon. This procedure
    is not required for multi-threaded daemons.

    \subsubsection{clamav-config}
    Use \verb+clamav-config+ to check compilation information for libclamav.
    \begin{verbatim}
	$ clamav-config --libs
	-L/usr/local/lib -lz -lbz2 -lgmp -lpthread
	$ clamav-config --cflags
	-I/usr/local/include -g -O2
    \end{verbatim}

    \subsubsection{Example}
    You will find an example scanner application in the clamav source
    package (/example). Provided you have ClamAV already installed, execute
    the following to compile it:
    \begin{verbatim}
	gcc -Wall ex1.c -o ex1 -lclamav
    \end{verbatim}

    \subsection{CVD format}
    CVD (ClamAV Virus Database) is a digitally signed tarball containing
    one or more databases. The header is a 512-bytes long string with colon
    separated fields:
    \begin{verbatim}
ClamAV-VDB:build time:version:number of signatures:functionality
level required:MD5 checksum:digital signature:builder name:build time (sec)
    \end{verbatim}
    \verb+sigtool --info+ displays detailed information on CVD files:
    \begin{verbatim}
$ sigtool -i daily.cvd 
File: daily.cvd
Build time: 10 Mar 2008 10:45 +0000
Version: 6191
Signatures: 59084
Functionality level: 26
Builder: ccordes
MD5: 6e6e29dae36b4b7315932c921e568330
Digital signature: zz9irc9irupR3z7yX6J+OR6XdFPUat4HIM9ERn3kAcOWpcMFxq
Fs4toG5WJsHda0Jj92IUusZ7wAgYjpai1Nr+jFfXHsJxv0dBkS5/XWMntj0T1ctNgqmiF
+RLU6V0VeTl4Oej3Aya0cVpd9K4XXevEO2eTTvzWNCAq0ZzWNdjc
Verification OK.
    \end{verbatim}

    \subsection{Contributors}
    The following people contributed to our project in some way (providing
    patches, bug reports, technical support, documentation, good ideas...):
    \begin{itemize}
	\item Ian Abbott \email{<abbotti*mev.co.uk>}
	\item Clint Adams \email{<schizo*debian.org>}
	\item Sergey Y. Afonin \email{<asy*kraft-s.ru>}
	\item Robert Allerstorfer \email{<roal*anet.at>}
	\item Claudio Alonso \email{<cfalonso*yahoo.com>}
	\item Kevin Amorin \email{<kamorin*ccs.neu.edu>}
	\item Kamil Andrusz \email{<wizz*mniam.net>}
	\item Tayfun Asker \email{<tasker*metu.edu.tr>}
	\item Jean-Edouard Babin \email{<Jeb*jeb.com.fr>}
	\item Marc Baudoin \email{<babafou*babafou.eu.org>}
	\item Scott Beck \email{<sbeck*gossamer-threads.com>}
	\item Rolf Eike Beer \email{<eike*mail.math.uni-mannheim.de>}
	\item Rene Bellora \email{<rbellora*tecnoaccion.com.ar>}
	\item Carlo Marcelo Arenas Belon \email{<carenas*sajinet.com.pe>}
	\item Joseph Benden \email{<joe*thrallingpenguin.com>}
	\item Hilko Bengen \email{<bengen*vdst-ka.inka.de>}
	\item Hank Beatty \email{<hbeatty*starband.net>}
	\item Alexandre Biancalana \email{<ale*seudns.net>}
	\item Patrick Bihan-Faou \email{<patrick*mindstep.com>}
	\item Martin Blapp \email{<mb*imp.ch>}
	\item Dale Blount \email{<dale*velocity.net>}
	\item Serge van den Boom \email{<svdb*stack.nl>}
	\item Oliver Brandmueller \email{<ob*e-Gitt.NET>}
	\item Boguslaw Brandys \email{<brandys*o2.pl>}
	\item Igor Brezac \email{<igor*ipass.net>}
	\item Mike Brudenell \email{<pmb1*york.ac.uk>}
	\item Brian Bruns \email{<bruns*2mbit.com>}
	\item Len Budney \email{<lbudney*pobox.com>}
	\item Matt Butt \email{<mattb*cre8tiv.com>}
	\item Christopher X. Candreva \email{<chris*westnet.com>}
	\item Eric I. Lopez Carreon \email{<elopezc*technitrade.com>}
	\item Ales Casar \email{<casar*uni-mb.si>}
	\item Jonathan Chen \email{<jon+clamav*spock.org>}
	\item Andrey Cherezov \email{<andrey*cherezov.koenig.su>}
	\item Alex Cherney \email{<alex*cher.id.au>}
	\item Tom G. Christensen \email{<tgc*statsbiblioteket.dk>}
	\item Nicholas Chua \email{<nicholas*ncmbox.net>}
	\item Chris Conn \email{<cconn*abacom.com>}
	\item Christoph Cordes \email{<ib*precompiled.de>}
	\item Ole Craig \email{<olc*cs.umass.edu>}
	\item Eugene Crosser \email{<crosser*rol.ru>}
	\item Calin A. Culianu \email{<calin*ajvar.org>}
	\item Damien Curtain \email{<damien*pagefault.org>}
	\item Krisztian Czako \email{<slapic*linux.co.hu>}
	\item Diego d'Ambra \email{<da*softcom.dk>}
	\item Michael Dankov \email{<misha*btrc.ru>}
	\item Yuri Dario \email{<mc6530*mclink.it>}
	\item David \email{<djgardner*users.sourceforge.net>}
	\item Maxim Dounin \email{<mdounin*rambler-co.ru>}
	\item Alejandro Dubrovsky \email{<s328940*student.uq.edu.au>}
	\item James P. Dugal \email{<jpd*louisiana.edu>}
	\item Magnus Ekdahl \email{<magnus*debian.org>}
	\item Mehmet Ekiz \email{<ekizm*tbmm.gov.tr>}
	\item Jens Elkner \email{<elkner*linofee.org>}
	\item Fred van Engen \email{<fred*wooha.org>}
	\item Jason Englander \email{<jason*englanders.cc>}
	\item Oden Eriksson \email{<oeriksson*mandrakesoft.com>}
	\item Daniel Fahlgren \email{<fahlgren*ardendo.se>}
	\item Andy Fiddaman \email{<af*jeamland.org>}
	\item Edison Figueira Junior \email{<edison*brc.com.br>}
	\item David Ford \email{<david+cert*blue-labs.org>}
	\item Martin Forssen \email{<maf*appgate.com>}
	\item Brian J. France \email{<list*firehawksystems.com>}
	\item Free Oscar \email{<freeoscar*wp.pl>}
	\item Martin Fuxa \email{<yeti*email.cz>}
	\item Piotr Gackiewicz \email{<gacek*intertele.pl>}
	\item Jeremy Garcia \email{<jeremy*linuxquestions.org>}
	\item Dean Gaudet \email{<dean-clamav*arctic.org>}
	\item Michel Gaudet \email{<Michel.Gaudet*ehess.fr>}
	\item Philippe Gay \email{<ph.gay*free.fr>}
	\item Nick Gazaloff \email{<nick*sbin.org>}
	\item Geoff Gibbs \email{<ggibbs*hgmp.mrc.ac.uk>}
	\item Luca 'NERvOus' Gibelli \email{<nervous*nervous.it>}
	\item Scott Gifford \email{<sgifford*suspectclass.com>}
	\item Wieslaw Glod \email{<wkg*x2.pl>}
	\item Stephen Gran \email{<steve*lobefin.net>}
	\item Koryn Grant \email{<koryn*endace.com>}
	\item Matthew A. Grant \email{<grantma*anathoth.gen.nz>}
	\item Christophe Grenier \email{<grenier*cgsecurity.org>}
	\item Marek Gutkowski \email{<hobbit*core.segfault.pl>}
	\item Jason Haar \email{<Jason.Haar*trimble.co.nz>}
	\item Hrvoje Habjanic \email{<hrvoje.habjanic*zg.hinet.hr>}
	\item Michal Hajduczenia \email{<michalis*mat.uni.torun.pl>}
	\item Jean-Christophe Heger \email{<jcheger*acytec.com>}
	\item Martin Heinz \email{<Martin*hemag.ch>}
	\item Kevin Heneveld" \email{<kevin*northstar.k12.ak.us>}
	\item Anders Herbjornsen \email{<andersh*gar.no>}
	\item Paul Hoadley \email{<paulh*logixsquad.net>}
	\item Robert Hogan \email{<robert*roberthogan.net>}
	\item Przemyslaw Holowczyc \email{<doozer*skc.com.pl>}
	\item Thomas W. Holt Jr. \email{<twh*cohesive.net>}
	\item James F.  Hranicky \email{<jfh*cise.ufl.edu>}
	\item Douglas J Hunley \email{<doug*hunley.homeip.net>}
	\item Kurt Huwig \email{<kurt*iku-netz.de>}
	\item Andy Igoshin \email{<ai*vsu.ru>}
	\item Michal Jaegermann \email{<michal*harddata.com>}
	\item Christophe Jaillet \email{<christophe.jaillet*wanadoo.fr>}
	\item Jay \email{<sysop-clamav*coronastreet.net>}
	\item Stephane Jeannenot \email{<stephane.jeannenot*wanadoo.fr>}
	\item Per Jessen \email{<per*computer.org>}
	\item Dave Jones \email{<dave*kalkbay.co.za>}
	\item Jesper Juhl \email{<juhl*dif.dk>}
	\item Kamil Kaczkowski \email{<kamil*kamil.eisp.pl>}
	\item Alex Kah \email{<alex*narfonix.com>}
	\item Stefan Kaltenbrunner \email{<stefan*kaltenbrunner.cc>}
	\item Lloyd Kamara \email{<l.kamara*imperial.ac.uk>}
	\item Stefan Kanthak \email{<stefan.kanthak*fujitsu-siemens.com>}
	\item Kazuhiko \email{<kazuhiko*fdiary.net>}
	\item Jeremy Kitchen \email{<kitchen*scriptkitchen.com>}
	\item Tomasz Klim \email{<tomek*euroneto.pl>}
	\item Robbert Kouprie \email{<robbert*exx.nl>}
	\item Martin Kraft \email{<martin.kraft*fal.de>}
	\item Petr Kristof \email{<Kristof.P*fce.vutbr.cz>}
	\item Henk Kuipers \email{<henk*opensourcesolutions.nl>}
	\item Nigel Kukard \email{<nkukard*lbsd.net>}
	\item Eugene Kurmanin \email{<smfs*users.sourceforge.net>}
	\item Dr Andrzej Kurpiel \email{<akurpiel*mat.uni.torun.pl>}
	\item Mark Kushinsky \email{<mark*mdspc.com>}
	\item Mike Lambert \email{<lambert*jeol.com>}
	\item Thomas Lamy \email{<Thomas.Lamy*in-online.net>}
	\item Stephane Leclerc \email{<sleclerc*aliastec.net>}
	\item Marty Lee \email{<marty*maui.co.uk>}
	\item Dennis Leeuw \email{<dleeuw*made-it.com>}
	\item Martin Lesser \email{<admin-debian*bettercom.de>}
	\item Peter N Lewis \email{<peter*stairways.com.au>}
	\item Matt Leyda \email{<mfleyda*e-one.com>}
	\item James Lick \email{<jlick*drivel.com>}
	\item Jerome Limozin \email{<jerome*limozin.net>}
	\item Mike Loewen \email{<mloewen*sturgeon.cac.psu.edu>}
	\item Roger Lucas \email{<roger*planbit.co.uk>}
	\item David Luyer \email{<david\_luyer*pacific.net.au>}
	\item Richard Lyons \email{<frob-clamav*webcentral.com.au>}
	\item David S. Madole \email{<david*madole.net>}
	\item Thomas Madsen \email{<tm*softcom.dk>}
	\item Bill Maidment \email{<bill*maidment.com.au>}
	\item Joe Maimon \email{<jmaimon*ttec.com>}
	\item David Majorel \email{<dm*lagoon.nc>}
	\item Andrey V. Malyshev \email{<amal*krasn.ru>}
	\item Fukuda Manabu \email{<fukuda*cri-mw.co.jp>}
	\item Stefan Martig \email{<sm*officeco.ch>}
	\item Alexander Marx \email{<mad-ml*madness.at>}
	\item Andreas Marx (\url{http://www.av-test.org/})
	\item Chris Masters \email{<cmasters*insl.co.uk>}
	\item Fletcher Mattox \email{<fletcher*cs.utexas.edu>}
	\item Serhiy V. Matveyev \email{<matveyev*uatele.com>}
	\item Reinhard Max \email{<max*suse.de>}
	\item Brian May \email{<bam*debian.org>}
	\item Ken McKittrick \email{<klmac*usadatanet.com>}
	\item Chris van Meerendonk \email{<cvm*castel.nl>}
	\item Andrey J. Melnikoff \email{<temnota*kmv.ru>}
	\item Damian Menscher \email{<menscher*uiuc.edu>}
	\item Denis De Messemacker \email{<ddm*clamav.net>}
	\item Jasper Metselaar \email{<jasper*formmailer.net>}
	\item Arkadiusz Miskiewicz \email{<misiek*pld-linux.org>}
	\item Ted Mittelstaedt \email{<tedm*toybox.placo.com>}
	\item Mark Mielke \email{<mark*mark.mielke.cc>}
	\item John Miller \email{<contact*glideslopesoftware.co.uk>}
	\item Jo Mills \email{<Jonathan.Mills*frequentis.com>}
	\item Dustin Mollo \email{<dustin.mollo*sonoma.edu>}
	\item Remi Mommsen \email{<remigius.mommsen*cern.ch>}
	\item Doug Monroe \email{<doug*planetconnect.com>}
	\item Alex S Moore \email{<asmoore*edge.net>}
	\item Tim Morgan \email{<tim*sentinelchicken.org>}
	\item Dirk Mueller \email{<mueller*kde.org>}
	\item Flinn Mueller\email{<flinn*activeintra.net>}
	\item Hendrik Muhs \email{<Hendrik.Muhs*student.uni-magdeburg.de>}
	\item Simon Munton \email{<simon*munton.demon.co.uk>}
	\item Farit Nabiullin (\url{http://program.farit.ru/})
	\item Nemosoft Unv. \email{<nemosoft*smcc.demon.nl>}
	\item Wojciech Noworyta \email{<wnow*konarski.edu.pl>}
	\item Jorgen Norgaard \email{<jnp*anneli.dk>}
	\item Fajar A. Nugraha \email{<fajar*telkom.co.id>}
	\item Joe Oaks \email{<joe.oaks*hp.com>}
	\item Washington Odhiambo \email{<wash*wananchi.com>}
	\item Masaki Ogawa \email{<proc*mac.com>}
	\item John Ogness \email{<jogness*antivir.de>}
	\item Phil Oleson \email{<oz*nixil.net>}
	\item Jan Ondrej \email{<ondrejj*salstar.sk>}
	\item Martijn van Oosterhout \email{<kleptog*svana.org>}
	\item OpenAntiVirus Team (\url{http://www.OpenAntiVirus.org/})
	\item Tomasz Papszun \email{<tomek*lodz.tpsa.pl>}
	\item Eric Parsonage \email{<eric*eparsonage.com>}
	\item Oliver Paukstadt \email{<pstadt*stud.fh-heilbronn.de>}
	\item Christian Pelissier \email{<Christian.Pelissier*onera.fr>}
	\item Rudolph Pereira \email{<rudolph*usyd.edu.au>}
	\item Dennis Peterson \email{<dennispe*inetnw.com>}
	\item Ed Phillips \email{<ed*UDel.Edu>}
	\item Andreas Piesk \email{<Andreas.Piesk*heise.de>}
	\item Mark Pizzolato \email{<clamav-devel*subscriptions.pizzolato.net>}
	\item Dean Plant \email{<dean.plant*roke.co.uk>}
	\item Alex Pleiner \email{<pleiner*zeitform.de>}
	\item Ant La Porte \email{<ant*dvere.net>}
	\item Jef Poskanzer \email{<jef*acme.com>}
	\item Christophe Poujol \email{<Christophe.Poujol*atosorigin.com>}
	\item Sergei Pronin \email{<sp*finndesign.fi>}
	\item Thomas Quinot \email{<thomas*cuivre.fr.eu.org>}
	\item Ed Ravin \email{<eravin*panix.com>}
	\item Robert Rebbun \email{<robert*desertsurf.com>}
	\item Brian A. Reiter \email{<breiter*wolfereiter.com>}
	\item Didi Rieder \email{<adrieder*sbox.tugraz.at>}
	\item Pavel V. Rochnyack \email{<rpv*fsf.tsu.ru>}
	\item Rupert Roesler-Schmidt \email{<r.roesler-schmidt*uplink.at>}
	\item David Sanchez \email{<dsanchez*veloxia.com>}
	\item David Santinoli \email{<david*santinoli.com>}
	\item Vijay Sarvepalli \email{<vssarvep*office.uncg.edu>}
	\item Martin Schitter
	\item Theo Schlossnagle \email{<jesus*omniti.com>}
	\item Enrico Scholz \email{<enrico.scholz*informatik.tu-chemnitz.de>}
	\item Karina Schwarz \email{<k.schwarz*uplink.at>}
	\item Scsi \email{<scsi*softland.ru>}
	\item Dr Matthew J Seaman \email{<m.seaman*infracaninophile.co.uk>}
	\item Hector M. Rulot Segovia \email{<Hector.Rulot*uv.es>}
	\item Omer Faruk Sen \email{<ofsen*enderunix.org>}
	\item Sergey \email{<a\_s\_y*sama.ru>}
	\item Tuomas Silen \email{<tuomas.silen*nodeta.fi>}
	\item David F. Skoll \email{<dfs*roaringpenguin.com>}
	\item Al Smith \email{<ajs+clamav*aeschi.ch.eu.org>}
	\item Sergey Smitienko \email{<hunter*comsys.com.ua>}
	\item Solar Designer \email{<solar*openwall.com>}
	\item Joerg Sonnenberger \email{<joerg*britannica.bec.de>}
	\item Michal 'GiM' Spadlinski (\url{http://gim.org.pl/})
	\item Kevin Spicer \email{<kevin*kevinspicer.co.uk>}
	\item GertJan Spoelman \email{<cav*gjs.cc>}
	\item Ole Stanstrup \email{<ole*stanstrup.dk>}
	\item Adam Stein \email{<adam*scan.mc.xerox.com>}
	\item Steve \email{<steveb*webtribe.net>}
	\item Richard Stevenson \email{<richard*endace.com>}
	\item Sven Strickroth \email{<sstrickroth*gym-oha.de>}
	\item Matt Sullivan \email{<matt*sullivan.gen.nz>}
	\item Dr Zbigniew Szewczak \email{<zssz*mat.uni.torun.pl>}
	\item Joe Talbott \email{<josepht*cstone.net>}
	\item Gernot Tenchio \email{<g.tenchio*telco-tech.de>}
	\item Masahiro Teramoto \email{<markun*onohara.to>}
	\item Daniel Theodoro \email{<dtheodoro*ig.com.br>}
	\item Ryan Thompson \email{<clamav*sasknow.com>}
	\item Gianluigi Tiesi \email{<sherpya*netfarm.it>}
	\item Yar Tikhiy \email{<yar*comp.chem.msu.su>}
	\item Andrew Toller \email{<atoller*connectfree.co.uk>}
	\item Michael L. Torrie \email{<torriem*chem.byu.edu>}
	\item Trashware \email{<trashware*gmx.net>}
	\item Matthew Trent \email{<mtrent*localaccess.com>}
	\item Reini Urban \email{<rurban*x-ray.at>}
	\item Daniel Mario Vega \email{<dv5a*dc.uba.ar>}
	\item Denis Vlasenko \email{<vda*ilport.com.ua>}
	\item Laurent Wacrenier \email{<lwa*teaser.fr>}
	\item Charlie Watts \email{<cewatts*brainstorminternet.net>}
	\item Florian Weimer \email{<fw*deneb.enyo.de>}
	\item Paul Welsh \email{<paul*welshfamily.com>}
	\item Nicklaus Wicker \email{<n.wicker*cnk-networks.de>}
	\item David Woakes \email{<david*mitredata.co.uk>}
	\item Troy Wollenslegel \email{<troy*intranet.org>}
	\item ST Wong \email{<st-wong*cuhk.edu.hk>}
	\item Dale Woolridge \email{<dwoolridge*drh.net>}
	\item David Wu \email{<dyw*iohk.com>}
	\item Takumi Yamane \email{<yamtak*b-session.com>}
	\item Youza Youzovic \email{<youza*post.cz>}
	\item Anton Yuzhaninov \email{<citrin*rambler-co.ru>}
	\item Leonid Zeitlin \email{<lz*europe.com>}
	\item ZMan Z. \email{<x86zman*go-a-way.dyndns.org>}
	\item Andoni Zubimendi \email{<andoni*lpsat.net>}
    \end{itemize}

    \subsection{Donors}
    We've received financial support from: (in alphabetical order)
    \begin{itemize}
	\item ActiveIntra.net Inc. (\url{http://www.activeintra.net/})
	\item Advance Healthcare Group (\url{http://www.ahgl.com.au/})
	\item Allied Quotes (\url{http://www.AlliedQuotes.com /})
	\item American Computer \& Electronic Services Corp. (\url{http://www.acesnw.com/})
	\item Amnesty International, Swiss Section (\url{http://www.amnesty.ch/})
	\item Steve Anderson
	\item Anonymous donor from Colorado, US
	\item Arudius (\url{http://arudius.sourceforge.net/})
	\item Peter Ashman
	\item Atlas College (\url{http://www.atlascollege.nl/})
	\item Australian Payday Cash Loans (\url{http://www.cashdoctors.com.au/})
	\item AWD Online (\url{http://www.awdonline.com/})
	\item BackupAssist Backup Software (\url{http://www.backupassist.com/})
	\item Dave Baker
	\item Bear and Bear Consulting, Inc. (\url{http://www.bear-consulting.com/})
	\item Aaron Begley
	\item Craig H. Block
	\item Norman E. Brake, Jr.
	\item Josh Burstyn
	\item By Design (\url{http://www.by-design.net/})
	\item Canadian Web Hosting (\url{http://www.canadianwebhosting.com/})
	\item cedarcreeksoftware.com (\url{http://www.cedarcreeksoftware.com/})
	\item Ricardo Cerqueira
	\item Thanos Chatziathanassiou
	\item Cheahch from Singapore
	\item Conexim Australia - business web hosting (\url{http://www.conexim.com.au})
	\item Alan Cook
	\item Joe Cooper
	\item CustomLogic LLC (\url{http://www.customlogic.com/})
	\item Ron DeFulio
	\item Digirati (\url{http://oss.digirati.com.br/})
	\item Steve Donegan (\url{http://www.donegan.org/})
	\item Dynamic Network Services, Inc (\url{http://www.dyndns.org/})
	\item EAS Enterprises LLC
	\item eCoupons.com (\url{http://www.ecoupons.com/})
	\item Electric Embers (\url{http://electricembers.net})
	\item John T. Ellis
	\item Epublica
	\item Bernhard Erdmann
	\item David Eriksson (\url{http://www.2good.nu/})
	\item Philip Ershler
	\item Explido Software USA Inc. (\url{http://www.explido.us/})
	\item David Farrick
	\item Jim Feldman
	\item Petr Ferschmann (\url{http://petr.ferschmann.cz/})
	\item Andries Filmer (\url{http://www.netexpo.nl/})
	\item The Free Shopping Cart people (\url{http://www.precisionweb.net/})
	\item Paul Freeman
	\item Jack Fung
	\item Stephen Gageby
	\item Paolo Galeazzi
	\item GANDI (\url{http://www.gandi.net/})
	\item Jeremy Garcia (\url{http://www.linuxquestions.org/})
	\item GBC Internet Service Center GmbH (\url{http://www.gbc.net/})
	\item GCS Tech (\url{http://www.gcstech.net/})
	\item GHRS (\url{http://www.ghrshotels.com/})
	\item Lyle Giese
	\item Todd Goodman
	\item Bill Gradwohl (\url{http://www.ycc.com/})
	\item Grain-of-Salt Consulting
	\item Terje Gravvold
	\item Hart Computer (\url{http://www.hart.co.jp/})
	\item Pen Helm
	\item Hosting Metro LLC (\url{http://www.hostingmetro.com/})
	\item IDEAL Software GmbH (\url{http://www.IdealSoftware.com/})
	\item Industry Standard Computers (\url{http://www.ISCnetwork.com/})
	\item Interact2Day (\url{http://www.interact2day.com/})
	\item Invisik Corporation (\url{http://www.invisik.com/})
	\item itXcel Internet - Domain Registration (\url{http://www.itxcel.com})
	\item Craig Jackson
	\item Stuart Jones
	\item Jason Judge
	\item Keith (\url{http://www.textpad.com/})
	\item Ewald Kicker (\url{http://www.very-clever.com/})
	\item Brad Koehn
	\item Christina Kuratli (\url{http://www.virusprotect.ch/})
	\item Logic Partners Inc. (\url{http://www.logicpartners.com/})
	\item Mark Lotspaih (\url{http://www.lotcom.org/})
	\item Michel Machado (\url{http://oss.digirati.com.br/})
	\item Olivier Marechal
	\item Matthew McKenzie
	\item Durval Menezes (\url{http://www.durval.com.br/})
	\item Micro Logic Systems (\url{http://www.mls.nc/})
	\item Midcoast Internet Solutions
	\item Mimecast (\url{http://www.mimecast.com/})
	\item Kazuhiro Miyaji
	\item Bozidar Mladenovic
	\item Paul Morgan
	\item Tomas Morkus
	\item The Names Database (\url{http://static.namesdatabase.com})
	\item Names Directory (\url{http://www.namesdir.com/})
	\item Michael Nolan (\url{http://www.michaelnolan.co.uk/})
	\item Jorgen Norgaard
	\item Numedeon, Inc. creators of Whyville (\url{http://www.whyville.net/})
	\item Oneworkspace.com (\url{http://www.oneworkspace.com/})
	\item Online Literature (\url{http://www.couol.com/})
	\item Origin Solutions (\url{http://www.originsolutions.com.au/})
	\item outermedia GmbH (\url{http://www.outermedia.de/})
	\item Kevin Pang (\url{http://www.freebsdblog.org/})
	\item Alexander Panzhin
	\item Passageway Communications (\url{http://www.passageway.com})
	\item Dan Pelleg (\url{http://www.libagent.org/})
	\item Thodoris Pitikaris
	\item Paul Rantin
	\item Thomas J. Raef (\url{http://www.ebasedsecurity.com})
	\item Luke Reeves (\url{http://www.neuro-tech.net/})
	\item RHX (\url{http://www.rhx.it/})
	\item Stefano Rizzetto
	\item Roaring Penguin Software Inc. (\url{http://www.roaringpenguin.com/})
	\item Luke Rosenthal
	\item Jenny S�fstr�m (\url{http://PokerListings.com})
	\item School of Engineering, University of Pennsylvania (\url{http://www.seas.upenn.edu/})
	\item Tim Scoff
	\item Seattle Server (\url{http://www.seattleserver.com/})
	\item Software Workshop Inc (\url{http://www.softwareworkshop.com/})
	\item Solutions In A Box (\url{http://www.siab.com.au/})
	\item Stephane Rault
	\item SearchMain (\url{http://www.searchmain.com/})
	\item Olivier Silber
	\item Fernando Augusto Medeiros Silva (\url{http://www.linuxplace.com.br/})
	\item Sollentuna Fria Gymnasium, Sweden (\url{http://www.sfg.se/})
	\item StarBand (\url{http://www.starband.com/})
	\item Stroke of Color, Inc.
	\item Synchro Sistemas de Informacao (\url{http://synchro.com.br/})
	\item Sahil Tandon
	\item The Spamex Disposable Email Address Service (\url{http://www.spamex.com})
	\item Brad Tarver
	\item TGT Tampermeier \& Grill Steuerberatungs- und Wirtschaftstreuhand OEG (\url{http://www.tgt.at/})
	\item Per Reedtz Thomsen
	\item William Tisdale
	\item Up Time Technology (\url{http://www.uptimetech.com/})
	\item Ulfi
	\item Jeremy Vanderburg (\url{http://www.jeremytech.com/})
	\item Web.arbyte - Online-Marketing (\url{http://www.webarbyte.de/})
	\item Webzone Srl (\url{http://www.webzone.it/})
	\item Markus Welsch (\url{http://www.linux-corner.net/})
	\item Julia White (\url{http://www.convert-tools.com/})
	\item Nicklaus Wicker
	\item David Williams (\url{http://kayakero.net/})
	\item Glenn R Williams
	\item Kelly Williams
	\item XRoads Networks (\url{http://xroadsnetworks.com/})
	\item Zimbra open-source collaboration suite (\url{http://www.zimbra.com/})
    \end{itemize}

    \subsection{Graphics}
    The ClamAV logo was created by Mia Kalenius and Sergei Pronin from
    Finndesign (\url{http://www.finndesign.fi/}).

    \subsection{OpenAntiVirus}
    Our database includes the virus database (about 7000 signatures) from
    OpenAntiVirus (\url{http://OpenAntiVirus.org}).

    \section{Core Team}

    \begin{itemize}
	\item Joel Esler \email{<jesler*cisco.com>}, USA\\
	Role: community manager

	\item Erin Germ \email{<egerm*cisco.com>}, USA\\
	Role: ClamAV quality engineering

	\item Douglas Gastonguay-Goddard \email{<douggg*cisco.com>}, USA\\
	Role: virus database maintainer

	\item Tom Judge \email{<tomjudge*cisco.com>}, USA\\
	Role: infrastucture developer

	\item Steven Morgan \email{<stevmorg*cisco.com>}, USA\\
	Role: ClamAV technical lead

	\item Matthew Olney \email{<molney*cisco.com>}, USA\\
	Role: development manager

	\item David Raynor \email{<draynor*cisco.com>}, USA\\
	Role: ClamAV developer

	\item Shawn Webb \email{<shawebb*sourcefire.com>}, USA\\
	Role: ClamAV developer

	\item Kevin Lin \email{<kevlin2*cisco.com>}, USA\\
	Role: ClamAV developer

	\item Dave Suffling \email{<dsufflin*cisco.com>}, Canada\\
	Role: ClamAV developer

	\item Samir Sapra \email{<ssapra*cisco.com>}, USA\\
	Role: ClamAV developer

	\item Alain Zidouemba \email{<azidouem*cisco.com>}, USA\\
	Role: manager, virus databases
      
    \end{itemize}

    \section{Emeritus Team}

    \begin{itemize}
	\item aCaB \email{<acab*clamav.net>}, Italy\\
	Role: virus database maintainer, coder

	\item Christoph Cordes \email{<ccordes*clamav.net>}, Germany\\
	Role: virus database maintainer

	\item Mike Cathey \email{<mike*clamav.net>}, USA\\
	Role: co-sysadmin

	\item Diego d'Ambra \email{<diego*clamav.net>}, Denmark\\
	Role: virus database maintainer

	\item Luca Gibelli \email{<luca*clamav.net>}, Italy\\
	Role: sysadmin, mirror coordinator

	\item Nigel Horne \email{<njh*clamav.net>}, United Kingdom\\
	Role: coder

	\item Arnaud Jacques \email{<arnaud*clamav.net>}, France\\
	Role: virus database maintainer

	\item Tomasz Kojm \email{<tkojm*clamav.net>}, Poland\\
	Role: project leader, coder

	\item Tomasz Papszun \email{<tomek*clamav.net>}, Poland\\
	Role: various help

	\item Sven Strickroth \email{<sven*clamav.net>}, Germany\\
	Role: virus database maintainer, virus submission management

	\item Edwin Torok \email{<edwin*clamav.net>}, Romania\\
	Role: coder

	\item Trog \email{<trog*clamav.net>}, United Kingdom\\
	Role: coder
    \end{itemize}
\end{document}
